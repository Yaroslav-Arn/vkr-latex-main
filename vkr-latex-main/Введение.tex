\newsection
\centertocsection{ВВЕДЕНИЕ}

В современном мире информационных технологий, где компьютерные сети проникают во все сферы бизнеса и жизни, эффективное администрирование становится неотъемлемой частью успешного функционирования организаций. Однако, с увеличением масштабов бизнеса и географического разнообразия его структур, возникают новые трудности работы IT-специалистов. В этом контексте, удаленное администрирование компьютеров становится не просто дополнительной опцией, а необходимостью.

Удаленное администрирование позволяет IT-администраторам, без физического присутствия, эффективно управлять компьютерами на больших расстояниях. Данная функция становится крайне важной и необходимой в условиях удаленной работы, где сотрудники могут находиться в разных частях города, страны или даже континента.

Также стоит отметить что время становится одним из самых ценных ресурсов, наравне с техническими и человеческими. В ситуации, когда необходимо провести оперативные действия с компьютером, даже краткая задержка может оказать существенное воздействие на процессы бизнеса. Взаимодействие с данными или настройка системы, осуществляемые IT-администратором, порой вынуждают обычного пользователя останавливать свою работу. Тем самым создавая диссонанс в отточенный механизм корпоративных процессов. В свете этих обстоятельств, удаленное администрирование выступает важным инструментом, призванным не просто обеспечивать функциональность, но и минимизировать воздействие на оперативную деятельность. Воплощаясь в максимально незаметных процессах, оно позволяет пользователям беспрепятственно продолжать свои деловые задачи, не отвлекаясь на технические моменты.

Таким образом разработка приложения удаленного администрирования компьютеров становится актуальной задачей для разработчиков программного обеспечения. Такое приложение должно обеспечивать не только простоту и удобство в использовании, но и надежность, безопасность и высокую производительность. Кроме того, оно должно быть гибким и масштабируемым, способным адаптироваться к изменяющимся потребностям и условиям бизнеса.

\emph{Цель настоящей работы} – разработка приложение удаленного администрирования компьютеров организации для улучшения эффективности работы IT-администраторов. Для достижения поставленной цели необходимо решить \emph{следующие задачи:}
\begin{itemize}
\item провести анализ предметной области;
\item разработать концептуальную модель приложения;
\item спроектировать приложение;
\item реализовать приложение.
\end{itemize}

\emph{Структура и объем работы.} Отчет состоит из введения, 4 разделов основной части, заключения, списка использованных источников, 2 приложений. Текст выпускной квалификационной работы равен \formbytotal{page}{страниц}{е}{ам}{ам}.

\emph{Во введении} сформулирована цель работы, поставлены задачи разработки, описана структура работы, приведено краткое содержание каждого из разделов.

\emph{В первом разделе} на стадии описания технической характеристики предметной области приводится сбор информации о потребностях IT-администраторов и данные о компьютерах в организации.

\emph{Во втором разделе} на стадии технического задания приводятся требования к разрабатываемому приложению.

\emph{В третьем разделе} на стадии технического проектирования представлены проектные решения для приложения.

\emph{В четвертом разделе} приводится список классов и их методов, использованных при разработке приложения, производится тестирование разработанного приложения.

В заключении излагаются основные результаты работы, полученные в ходе разработки.

В приложении А представлен графический материал.
В приложении Б представлены фрагменты исходного кода. 
