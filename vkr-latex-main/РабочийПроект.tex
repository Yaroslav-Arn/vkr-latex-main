\newsection
\section{Рабочий проект}
\subsection{Спецификация компонентов и классов программной системы}

Для клиентской и серверной части приведено описание основных классов, их полей и методов. Фрагменты исходного кода приведены в приложении Б.

\subsubsection{Спецификация классов клиента}
Как уже говорилось в пункте 3.4.1.1 технического проекта клиентская часть программы содержит несколько классов. В таблицах \ref{class:fieldInI} - \ref{class:fieldView} приведено описание полей данных классов. 

\begin{longtable}[l]{|p{4.25cm}|p{2.35cm}|p{3.25cm}|p{4.85cm}|}
\caption{Описание полей класса IniFile\label{class:fieldInI}}\\
\hline \centrow Название поля & \centrow Метод
доступа & \centrow Тип данных & \centrow Описание \\
\hline \centrow 1 & \centrow 2 & \centrow 3 & \centrow 4\\
\endfirsthead
\caption*{Продолжение таблицы \ref{class:fieldInI}}\\
\hline \centrow 1 & \centrow 2 & \centrow 3 & \centrow 4\\
\finishhead
\hline  iniFilePath & private & string & Содержит полный путь и имя до INI файла\\
\hline IpComboBox & private & ComboBox & Содержитэкземпляр элемента ComboBox в который записываются доступные ip-адреса для вывода на форму 
\end{longtable}
\vspace{-\tablebelowskip}

\begin{longtable}[l]{|p{4.25cm}|p{2.35cm}|p{3.25cm}|p{4.85cm}|}
	\caption{Описание полей класса Client\label{class:fieldClient}}\\
	\hline \centrow Название поля & \centrow Метод
	доступа & \centrow Тип данных & \centrow Описание \\
	\hline \centrow 1 & \centrow 2 & \centrow 3 & \centrow 4\\
	\endfirsthead
	\caption*{Продолжение таблицы \ref{class:fieldClient}}\\
	\hline \centrow 1 & \centrow 2 & \centrow 3 & \centrow 4\\
	\finishhead
	\hline Ip & private & string & Строка в которую записывается текущий ip-адрес по которому будет осуществлено подключение к серверу\\
	\hline fileView & private & FileView & Экземпляр класса FileView\\
	\hline iniFile & private & IniFile & Экземпляр класса IniFile\\
	\hline tcpSocket & private & TcpSocket & Экземпляр класса TcpSocket\\
	\hline filePathClient & private & string & Строка которая используется для текущего выбранного пути в файловой системе на стороне клиента\\
	\hline filePathServer & private & string & Строка которая используется для текущего выбранного пути в файловой системе на стороне сервера
\end{longtable}
\vspace{-\tablebelowskip}

\begin{longtable}[l]{|p{4.25cm}|p{2.35cm}|p{3.25cm}|p{4.85cm}|}
	\caption{Описание полей класса TcpSocket\label{class:fieldTcp}}\\
	\hline \centrow Название поля & \centrow Метод
	доступа & \centrow Тип данных & \centrow Описание \\
	\hline \centrow 1 & \centrow 2 & \centrow 3 & \centrow 4\\
	\endfirsthead
	\caption*{Продолжение таблицы \ref{class:fieldTcp}}\\
	\hline \centrow 1 & \centrow 2 & \centrow 3 & \centrow 4\\
	\finishhead
	\hline client & private & TcpClient & Экземпляр класса TcpClient из пространства имен System.Net.Sockets, который построен поверх сокетов и использует сокет для отправки и получения данных, но при этом упрощает написание некоторых вещей\\
	\hline stream & private & NetworkStream & Экземпляр класса NetworkStream из пространства имен System.Net.Sockets, который используется для получения и отправки данных\\
	\hline ServerConsole-
	TextBox & private & TextBox & Экземпляр класса TextBox, элемент которого используется для вывода сообщений cmd полученных от сервера\\
	\hline fileView & private & FileView & Экземпляр класса FileView\\
	\hline Ip & private & string & Строка в которую записывается текущий ip-адрес по которому будет осуществлено подключение к серверу
\end{longtable}
\vspace{-\tablebelowskip}

\begin{longtable}[l]{|p{4.25cm}|p{2.35cm}|p{3.25cm}|p{4.85cm}|}
	\caption{Описание полей класса FileView\label{class:fieldView}}\\
	\hline \centrow Название поля & \centrow Метод
	доступа & \centrow Тип данных & \centrow Описание \\
	\hline \centrow 1 & \centrow 2 & \centrow 3 & \centrow 4\\
	\endfirsthead
	\caption*{Продолжение таблицы \ref{class:fieldView}}\\
	\hline \centrow 1 & \centrow 2 & \centrow 3 & \centrow 4\\
	\finishhead
	\hline ClientView & private & TreeView & Экземпляр класса TreeView, элемент которого используется для вывода файловой системы клиента\\
	\hline ServerView & private & TreeView & Экземпляр класса TreeView, элемент которого используется для вывода файловой системы сервера
\end{longtable}
\vspace{-\tablebelowskip}

В таблицах \ref{class:tableInI} - \ref{class:tableView} приведено описание методов данных классов. 

\begin{longtable}[l]{|p{4.25cm}|p{2.35cm}|p{8.1cm}|}
	\caption{Описание методов класса IniFile\label{class:tableInI}}\\
	\hline \centrow Название метода & \centrow Метод
	доступа & \centrow Описание \\
	\hline \centrow 1 & \centrow 2 & \centrow 3\\
	\endfirsthead
	\caption*{Продолжение таблицы \ref{class:tableInI}}\\
	\hline \centrow 1 & \centrow 2 & \centrow 3\\
	\finishhead
	\hline AddId & public & Метод который добавляет введенный пользователем ip-адрес.
	Результат: если ip-адреса не существует в INI файле, то он добавляется в файл, иначе выводит сообщение об ошибке.\\
	\hline IsValueExist & public & Метод проверки введенного пользователем ip-адреса на соответствие с имеющимися ip-адресами.
	Результат: возвращает true если совпадение есть, иначе false.\\
	\hline GenerateUniqueKey & private & Метод генерации уникального ключа для сохранения ip-адреса в INI файл.
	Результат: генерируется случайный ключ для сохранения ip-адреса\\
	\hline DeleteIp & public & Метод удаления введенного пользователем ip-адреса для удаления из INI файла. \\
	\hline LoadIpComboBoxItems & public & Метод считывания всех имеющихся в INI файле ip-адресов, а также вывода их в ComboBox.
	Результат: в ComboBox отображаются все имеющиеся ip-адреса. 
\end{longtable}
\vspace{-\tablebelowskip}

\begin{longtable}[l]{|p{4.25cm}|p{2.35cm}|p{8.1cm}|}
	\caption{Описание методов класса Client\label{class:tableClient}}\\
	\hline \centrow Название метода & \centrow Метод
	доступа & \centrow Описание \\
	\hline \centrow 1 & \centrow 2 & \centrow 3\\
	\endfirsthead
	\caption*{Продолжение таблицы \ref{class:tableClient}}\\
	\hline \centrow 1 & \centrow 2 & \centrow 3\\
	\finishhead
	\hline MainForm\_Resize & private & Метод обработки события изменения элемента управления. Используется для настройки элементов на форме интерфейса.
	Результат: при изменении формы также пропорционально ей меняются и все элементы.\\
	\hline ClientView
	\_BeforeExpand & private & Метод обработки события развертывания узла на элементе TreeView. Используется для вывода содержимого выбранной папки в файловой системе клиента.
	Результат: выводит список всех файлов и папок в выбранной ветке.\\
	\hline ServerView
	\_BeforeExpand & private & Метод обработки события развертывания узла на элементе TreeView. Используется для вывода содержимого выбранной папки в файловой системе сервера.
	Результат: выводит список всех файлов и папок в выбранной ветке.\\
	\hline ClientView
	\_AfterSelect & private & Метод обработки события изменения узла на элементе TreeView. Используется для отображения пути выбранного пользователем на стороне клиента.
	Результат: выводит выбранный пользователем путь.\\
	\hline ServerView
	\_AfterSelect & private & Метод обработки события изменения узла на элементе TreeView. Используется для отображения пути выбранного пользователем на стороне сервера.
	Результат: выводит выбранный пользователем путь.\\
	\hline FileButton\_Click & private & Метод обработки нажатия кнопки FileButton. Используется для перехода в режим работы с файловыми системами.
	Результат: скрываются все лишние элементы и открываются необходимые для работы.\\
	\hline ConsoleButton
	\_Click & private & Метод обработки нажатия кнопки ConsoleButton. Используется для перехода в режим работы с консолью.
	Результат: скрываются все лишние элементы и открываются необходимые для работы.\\
	\hline ConnectButton
	\_Click & private & Метод обработки нажатия кнопки ConnectButton. Используется для создания подключения к серверу по выбранному ip-адресу.
	Результат: считывает ip-адрес и соответствующего TextBox и запускает функцию установки соединения.\\
	\hline ClientConsoleText-
	Box\_KeyDown & private & Метод обработки нажатия кнопки Enter. Используется для отправки введенной команды на сервер.
	Результат: считывает команду и соответствующего TextBox и запускает функцию отправки команды.\\
	\hline DisconnectButton
	\_Click & private & Метод обработки нажатия кнопки DisconnectButton. Используется для разрыва текущего соединения с сервером.
	Результат: соединение с сервером разрывается.\\
	\hline AddButton\_Click & private & Метод обработки нажатия кнопки AddButton. Используется для сохранения ip-адреса.
	Результат: Считывает ip-адрес и соответствующего TextBox и запускает функцию сохранения его в INI файла.\\
	\hline SendButton\_Click & private &  Метод обработки нажатия кнопки SendButton. Используется для отправки файла на сервер.
	Результат: считывает файл по выбранному пути и путь по которому сохранить его на сервере и запускает функцию отправки файла на сервер.\\
	\hline ReceiveButton\_Click & private & Метод обработки нажатия кнопки ReceiveButton. Используется для получения файла от сервера.
	Результат: считывает путь к файлу, путь сохранения файла  и запускает функцию получения файла от сервер.\\
	\hline DeleteButton\_Click & private & Метод обработки нажатия кнопки DeleteButton. Используется для удаления ip-адреса.
	Результат: Считывает ip-адрес и соответствующего TextBox и запускает функцию удаление его из INI файла.
\end{longtable}
\vspace{-\tablebelowskip}

\begin{longtable}[l]{|p{4.25cm}|p{2.35cm}|p{8.1cm}|}
	\caption{Описание методов класса TcpSocket\label{class:tableTcp}}\\
	\hline \centrow Название метода & \centrow Метод
	доступа & \centrow Описание \\
	\hline \centrow 1 & \centrow 2 & \centrow 3\\
	\endfirsthead
	\caption*{Продолжение таблицы \ref{class:tableTcp}}\\
	\hline \centrow 1 & \centrow 2 & \centrow 3\\
	\finishhead
	\hline Start & private & Метод асинхронного запуска поиска сервера.
	Результат: запуск функции  ConnectToServerAsync \\
	\hline ConnectToServer-
	Async & private & Метод асинхронного подключения к указанному серверу на порте 8888.
	Результат: если предварительно сервер запущен, то осуществляется подключение, иначе выводится сообщение о соответствующей ошибке.\\
	\hline ReceiveMessage-
	Async & private & Метод получения и обработки полученных данных от сервера.
	Результат: в зависимости от типа сообщения запускает необходимые функции обработки входных данных.\\
	\hline SendCommand & public & Метод отправки команды на сервер.
	Результат: команда конвертируется в массив байт и по соответствующему протоколу отправляется на сервер.\\
	\hline SendDirectoryDrive & public & Метод отправки запроса на получения списка дисков на сервер.
	Результат: запрос конвертируется в массив байт и по соответствующему протоколу отправляется на сервер.\\
	\hline SendDirectoryFolder & public & Метод отправки запроса содержимого выбранной папки на сервер.
	Результат: запрос конвертируется в массив байт и по соответствующему протоколу отправляется на сервер.\\
	\hline SendFileAsync & public &  Метод отправки файла на сервер.
	Результат: запрос конвертируется в массив байт и по соответствующему протоколу отправляется на сервер.\\
	\hline SendReceiveFile & public & Метод запроса на получение файла на сервер.
	Результат: запрос конвертируется в массив байт и по соответствующему протоколу отправляется на сервер.\\
	\hline ReceiveDataAsync & private & Метод вычитки определенного количества байт из потока. Используется для считывания полученных данных в соответствии с протоколами.
	Результат: из потока вычитывается указанное количество байт\\
	\hline ReceiveDrivesAsync & private & Метод получения списка запрошенных дисков ПК сервера.
	Результат: полученный список дисков сервера конвертируется в строку для вывода.\\
	\hline ReceiveDirectory-
	Async & private & Метод получения списка содержимого запрошенной директории сервера.
	Результат: полученный список содержимого директории сервера конвертируется в строку для вывода.\\
	\hline ReceiveFileAsync & private & Метод получения файла от сервера.
	Результат: полученный файл сохраняется по указанному пути. 
\end{longtable}
\vspace{-\tablebelowskip}

\begin{longtable}[l]{|p{4.25cm}|p{2.35cm}|p{8.1cm}|}
	\caption{Описание методов класса FileView\label{class:tableView}}\\
	\hline \centrow Название метода & \centrow Метод
	доступа & \centrow Описание \\
	\hline \centrow 1 & \centrow 2 & \centrow 3\\
	\endfirsthead
	\caption*{Продолжение таблицы \ref{class:tableView}}\\
	\hline \centrow 1 & \centrow 2 & \centrow\\
	\finishhead
	\hline LoadDrives & private & Метод загрузки списка дисков ПК клиента.
	Результат: в TreeView выводятся все диски на ПК клиента\\
	\hline LoadFiles & public & Метод вывода списка содержимого выбранной папки на ПК клиента.
	Результат: в TreeView выводятся содержимое выбранной папки на ПК клиента\\
	\hline FindNodeRecursively & private & Метод поиска ранее открытых узлов TreeView.
	Результат: возвращает полный путь узла или null\\
	\hline FindDriveNode & public & Метод поиска ранее открытых дисков и подпапок.
	Результат: возвращает имя диска или подпапки или null\\
	\hline LoadSeverDrive & public & Метод вывода списка дисков ПК сервера.
	Результат: в TreeView выводятся все диски на ПК сервера\\
	\hline LoadServerFiles & public & Метод вывода списка содержимого выбранной папки ПК сервера.
	Результат: в TreeView выводятся содержимое выбранной папки ПК сервера\\
\end{longtable}
\vspace{-\tablebelowskip}

\subsubsection{Спецификация классов сервера}
Как уже говорилось в пункте 3.4.1.2 технического проекта серверная часть программы содержит один класс. В таблицах \ref{class:fieldServer} - \ref{class:tableServer} приведены описание полей и методов данного класса. 

\begin{longtable}[l]{|p{4cm}|p{2.35cm}|p{3.25cm}|p{5.1cm}|}
	\caption{Описание полей класса Server\label{class:fieldServer}}\\
	\hline \centrow Название поля & \centrow Метод
	доступа & \centrow Тип данных & \centrow Описание \\
	\hline \centrow 1 & \centrow 2 & \centrow 3 & \centrow 4\\
	\endfirsthead
	\caption*{Продолжение таблицы \ref{class:fieldServer}}\\
	\hline \centrow 1 & \centrow 2 & \centrow 3 & \centrow 4\\
	\finishhead
	\hline tcpListener & private & TcpListener & Экземпляр класса TcpListener из пространства имен System.Net.Sockets, который построен поверх сокетов и позволяет упростить написание некоторых вещей, по сравнению с использованием чистых сокетов.\\
	\hline process & private & Process & Экземпляр класса Process из пространства имен System.Diagnostics. Этот класс позволяет управлять уже запущенными процессами, а также запускать новые. 
\end{longtable}
\vspace{-\tablebelowskip}

\begin{longtable}[l]{|p{4.25cm}|p{2.35cm}|p{8.1cm}|}
	\caption{Описание методов класса Server\label{class:tableServer}}\\
	\hline \centrow Название метода & \centrow Метод
	доступа & \centrow Описание \\
	\hline \centrow 1 & \centrow 2 & \centrow 3\\
	\endfirsthead
	\caption*{Продолжение таблицы \ref{class:tableServer}}\\
	\hline \centrow 1 & \centrow 2 & \centrow 3\\
	\finishhead
	\hline InitializeServer & private & Метод асинхронного запуска сервера.
	Результат: вызов метода StartServerAsync.\\
	\hline StartServerAsync & private & Метод запуска сервера на порте 8888 и ожидания подключения клиента.
	Результат: ожидает подключение клиента. Если оно есть вызывается HandleClientCommAsync.\\
	\hline HandleClient-
	CommAsync & private & Метод ожидания и считывания полученных данных от клиента.
	Результат: вызывает метод StartCmdInBackground и в зависимости от полученного сообщения функцию обработки.\\
	\hline StartCmdIn-
	Background & private & Метод запуска сmd на ПК сервера.
	Результат: cmd запускается в фоновом режиме от имени администратора. \\
	\hline ExecuteCommand-
	Async & private & Метод выполнения полученной команды.
	Результат: отправляет команду в cmd на выполнение.\\
	\hline ReceiveDataAsync & private & Метод вычитки определенного количества байт из потока. Используется для считывания полученных данных в соответствии с протоколами.
	Результат: из потока вычитывается указанное количество байт\\
	\hline ReceiveFileAsync & private & Метод получения файла от клиента.
	Результат: полученный файл сохраняется по указанному пути. \\
	\hline SendCommandAsync & private & Метод отправки результата выполнения команды клиенту.
	Результат: команда в режиме реального времени транслируется клиенту.\\
	\hline SendDrivesAsync & private &  Метод отправки списка дисков клиенту.
	Результат: команда конвертируется в массив байт и по соответствующему протоколу отправляется клиенту.\\
	\hline SendFolderAsync & private & Метод отправки списка содержимого папки клиенту.
	Результат: список конвертируется в массив байт и по соответствующему протоколу отправляется клиенту.\\
	\hline GetDirectoryData & private & Метод считывания списка содержимого папки.
	Результат: возвращает список содержимого для отправки.\\
	\hline SendFileAsync & private & Метод отправки файла клиенту.
	Результат: файл конвертируется в массив байт и по соответствующему протоколу отправляется клиенту.
\end{longtable}
\vspace{-\tablebelowskip}

\subsubsection{Спецификация enum которые используются в программной системе}
Для того, чтобы в программе посылаемые сообщения не путались между собой были придуманы так называемые флаги. Они представляют из себя имена по которым клиент и сервер способны распознать какое именно сообщение приходит по потоку на данный момент. Реализована с помощью типа enum. В таблице \ref{class:tableEnum} приведены названия и описания этих флагов.

\begin{longtable}[l]{|p{4.25cm}|p{10.55cm}|}
	\caption{Описание флагов в enum\label{class:tableEnum}}\\
	\hline \centrow Название & \centrow Описание \\
	\hline \centrow 1 & \centrow 2 \\
	\endfirsthead
	\caption*{Продолжение таблицы \ref{class:tableEnum}}\\
	\hline \centrow 1 & \centrow 2 \\
	\finishhead
	\hline FilePath & Флаг который означает, что текущее сообщения передает файл и путь по которому его необходимо сохранить \\
	\hline Command & Флаг который означает, что текущее сообщение передает команду. её следует выполнить в cmd и отправить ответ клиенту\\
	\hline DirectoryDrive & Флаг который означает, что текущее сообщение является запросом на отправку списка дисков сервера клиенту\\
	\hline DirectoryFolder & Флаг который означает, что текущее сообщение является запросом на отправку содержимого папки клиенту по указанному пути\\
	\hline ReceiveFile & Флаг который означает, что текущее сообщение является запросом на отправку файла клиенту
\end{longtable}
\vspace{-\tablebelowskip}

\subsection{Системное тестирование разработанного приложения}

На рисунке \ref{1:image} представлен интерфейс программы на стороне клиента. При запуске воспроизводится вариант использования консоли сервера.

\begin{figure}
	\centering
	\includegraphics[width=1\linewidth]{"images/1"}
	\caption{Окно при запуске программы}
	\label{1:image}
\end{figure}

На рисунке \ref{2:image} представлено динамичное изменение интерфейса при деформации окна программы.

\begin{figure}
	\centering
	\includegraphics[width=1\linewidth]{"images/2"}
	\caption{Изменение интерфейса при деформации окна программы}
	\label{2:image}
\end{figure}

На рисунке \ref{3:image} представлено добавление нового ip-адреса в INI файл программы.

\begin{figure}
	\centering
	\includegraphics[width=1\linewidth]{"images/3"}
	\caption{Добавление нового ip-адреса}
	\label{3:image}
\end{figure}

На рисунке \ref{4:image} представлен запуск серверной части программной системы в виде службы.

\begin{figure}
	\centering
	\includegraphics[width=1\linewidth]{"images/4"}
	\caption{Запуск серверной части системы в виде службы}
	\label{4:image}
\end{figure}

На рисунке \ref{5:image} представлен процесс успешного подключения к серверу по ранее добавленному ip-адресу.

\begin{figure}
	\centering
	\includegraphics[width=1\linewidth]{"images/5"}
	\caption{Подключение к серверу}
	\label{5:image}
\end{figure}

На рисунке \ref{6:image} представлен процесс отправки команды в cmd сервера и получение результата её выполнения.

\begin{figure}
	\centering
	\includegraphics[width=1\linewidth]{"images/6"}
	\caption{Отправка команды и получения результата её выполнения}
	\label{6:image}
\end{figure}

На рисунке \ref{7:image} представлено переключения в режим работы файловой системы ПК.
\begin{figure}
	\centering
	\includegraphics[width=1\linewidth]{"images/7"}
	\caption{Переключение в режим работы с файловой системой}
	\label{7:image}
\end{figure}

На рисунке \ref{8:image} представлено переключения в режим работы файловой системы при существующем активном соединении.

\begin{figure}
	\centering
	\includegraphics[width=1\linewidth]{"images/8"}
	\caption{Переключение в режим работы с файловой системой при существующем соединении}
	\label{8:image}
\end{figure}

На рисунке \ref{9:image} представлен процесс просмотра содержимого дисков и папок обоих ПК. А также процесс выбора пути для получения или отправления файлов.

\begin{figure}
	\centering
	\includegraphics[width=1\linewidth]{"images/9"}
	\caption{Процесс просмотра содержимого дисков и папок}
	\label{9:image}
\end{figure}

На рисунке \ref{10:image} представлен процесс отправки файла на сервер.

\begin{figure}
	\centering
	\includegraphics[width=1\linewidth]{"images/10"}
	\caption{Отправка файла на сервер}
	\label{10:image}
\end{figure}

На рисунке \ref{11:image} представлен процесс получения файла от сервера

\begin{figure}
	\centering
	\includegraphics[width=1\linewidth]{"images/11"}
	\caption{Получение файла от сервера}
	\label{11:image}
\end{figure}

На рисунке \ref{12:image} представлен процесс отключения клиента от сервера.

\begin{figure}
	\centering
	\includegraphics[width=1\linewidth]{"images/12"}
	\caption{Отключение клиента от сервера}
	\label{12:image}
\end{figure}

На рисунке \ref{13:image} представлен процесс удаления ip-адреса из INI файла.

\begin{figure}
	\centering
	\includegraphics[width=1\linewidth]{"images/13"}
	\caption{Удаления ip-адреса из INI файла}
	\label{13:image}
\end{figure}

На рисунке \ref{14:image} представлен результат попытки подключения к не работающему или выключенному серверу.

\begin{figure}
	\centering
	\includegraphics[width=1\linewidth]{"images/14"}
	\caption{Ошибка при подключении к серверу}
	\label{14:image}
\end{figure}

На рисунке \ref{15:image} представлен результат попытки удалить ip-адрес во время существования активного подключения.
\begin{figure}
	\centering
	\includegraphics[width=1\linewidth]{"images/15"}
	\caption{Ошибка при удалении ip-адреса}
	\label{15:image}
\end{figure}

На рисунке \ref{16:image} представлен результат попытки добавления повторяющегося ip-адреса.

\begin{figure}
	\centering
	\includegraphics[width=1\linewidth]{"images/16"}
	\caption{Ошибка добавления ip-адреса}
	\label{16:image}
\end{figure}

\subsection{Сборка программной системы}

Для компиляции и сборки всех компонентов, входящих в состав
программно-информационной системы использовалась Visual Studio 2022.
Программную систему можно запустить в редакторе Visual Studio.
Программный продукт можно запустить на операционной системе Windows.