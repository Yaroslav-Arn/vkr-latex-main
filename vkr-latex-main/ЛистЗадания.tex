\newsection
\begin{center}
\large\textbf{Минобрнауки России}

\large\textbf{Юго-Западный государственный университет}
\vskip 1em
\normalsize{Кафедра программной инженерии}
\vskip 1em

\begin{flushright}
\begin{tabular}{p{.4\textwidth}}
\centrow УТВЕРЖДАЮ: \\
\centrow Заведующий кафедрой \\
\hrulefill \\
\setarstrut{\footnotesize}
\centrow\footnotesize{(подпись, инициалы, фамилия)}\\
\restorearstrut
«\underline{\hspace{1cm}}»
\underline{\hspace{3cm}}
20\underline{\hspace{1cm}} г.\\
\end{tabular}
\end{flushright}
\end{center}
\section*{ЗАДАНИЕ НА ВЫПУСКНУЮ КВАЛИФИКАЦИОННУЮ РАБОТУ
  ПО ПРОГРАММЕ БАКАЛАВРИАТА}
{\parindent0pt
  Студента \АвторРод, шифр\ \Шифр, группа \Группа
  
1. Тема «\Тема\ \ТемаВтораяСтрока» утверждена приказом ректора ЮЗГУ от \ДатаПриказа\ № \НомерПриказа.

2. Срок предоставления работы к защите \СрокПредоставления

3. Исходные данные для создания программной системы:

3.1. Перечень решаемых задач:}

\begin{enumerate}[label=\arabic*)]
\item проанализировать IT-инфраструктуру организации;
\item  разработать концептуальную модель системы управления IT-ин\-фра\-струк\-турой предприятия на основе подхода к управлению и расположению компьютеров в организации;
\item спроектировать программную систему удалённого администрирования IT-ин\-фра\-струк\-турой организации;
\item сконструировать и протестировать программную систему удалённого администрирования компьютеров организации.
\end{enumerate}

{\parindent0pt
  3.2. Входные данные и требуемые результаты для программы:}

\begin{enumerate}[label=\arabic*)]
\item Входными данными для программной системы являются: данные по IP адресам компьютеров организации, введённые команды, данные о подключении и отключение от компьютера.
\item Выходными данными для программной системы являются: ответ компьютера-сервера на подключение или отключение, ответ на введённую выполненную команду.
\end{enumerate}

{\parindent0pt

  4. Содержание работы (по разделам):
  
  4.1. Введение
  
  4.2. Анализ предметной области
  
4.3. Техническое задание: основание для разработки, назначение разработки,
требования к программной системе, требования к оформлению документации.

4.4. Технический проект: общие сведения о программной системе, проект
данных программной системы, проектирование архитектуры программной системы, проектирование пользовательского интерфейса программной системы.

4.5. Рабочий проект: спецификация компонентов и классов программной системы, тестирование программной системы, сборка компонентов программной системы.

4.6. Заключение

4.7. Список использованных источников

5. Перечень графического материала:

\begin{enumerate}[label=Лист \arabic*.]
\item Сведения о ВКРБ
\item Цель и задачи разработки
\item Диаграмма архитектуры локальной сети
\item Диаграмма прецедентов
\item UML диаграмма логики действия приложения
\item Диаграмма классов клиента
\item Диаграмма классов сервера
\item Заключение
\end{enumerate}

\vskip 2em
\begin{tabular}{@{}p{6.8cm}C{3.8cm}C{4.8cm}}
Руководитель ВКР & \lhrulefill{\fill} & \fillcenter\Руководитель\\
\setarstrut{\footnotesize}
& \footnotesize{(подпись, дата)} & \footnotesize{(инициалы, фамилия)}\\
\restorearstrut
Задание принял к исполнению & \lhrulefill{\fill} & \fillcenter\Автор\\
\setarstrut{\footnotesize}
& \footnotesize{(подпись, дата)} & \footnotesize{(инициалы, фамилия)}\\
\restorearstrut
\end{tabular}
}
