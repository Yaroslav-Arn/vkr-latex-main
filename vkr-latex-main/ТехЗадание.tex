\newsection
\section{Техническое задание}
\subsection{Основание для разработки}
Основанием для разработки программно-информационной системы для удаленного администрирования компьютеров организации является задание на выпускную квалификационную работу приказ ректора ЮЗГУ от <<  >>     2024 года № 0000-0 <<Об утверждении тем выпускных квалификационных работ и руководителей выпускных квалификационных работ>>.

\subsection{Цель и назначение разработки}
Целью является разработка приложения удаленного администрирования компьютеров организации для улучшения эффективности работы IT-администраторов. 

Функциональное назначение разрабатываемого приложения заключается в предоставлении IT-администраторам организации удаленного доступа к консоли и файловой системе компьютеров сотрудников компании.

Предполагается, что данной программой будут пользоваться работники IT-отдела организации для облегчения выполнения своих обязанностей и сокращения затрат человека-часов на решения возникших проблем.
На рисунке \ref{fig:-arx_diagramma} представлена диаграмма архитектуры локальной сети организации. На ней изображен пример того, как выглядит компьютерная сеть для которой необходимо данное приложение.

\begin{figure}
	\centering
	\includegraphics[width=1\linewidth]{"images/Диаграмма архитектуры"}
	\caption{Диаграмма архитектуры локальной сети}
	\label{fig:-arx_diagramma}
\end{figure}

Задачами разработки данного приложения являются:
\begin{itemize}
	\item создание клиентской части приложения;
	\item создание серверной части приложения;
	\item реализация клиент-серверного соединения;
	\item реализация обмена данными.
\end{itemize}

На рисунке \ref{fig:-action_diagramma} представлена UML диаграмма логики действия приложения. На ней видно как будет проходить процесс передачи пакетов между клиентом и сервером.
\begin{figure}
	\centering
	\includegraphics[width=1\linewidth]{"images/UML"}
	\caption{UML диаграмма логики действия приложения}
	\label{fig:-action_diagramma}
\end{figure}

\subsection{Требования пользователя к программному продукту}
\subsubsection{Требования к данным программного продукта}
Входными данными для приложения являются:
\begin{itemize}
	\item ip-адрес компьютера пользователя;
	\item команда для выполнения cmd;
	\item пути к папкам для отправки/получения файла;
	\item пути к файлам для отправки/получения.
\end{itemize}

Выходными данными для приложения являются:
\begin{itemize}
	\item файловая структура клиента;
	\item файловая структура сервера;
	\item результат выполнения команды в cmd.
\end{itemize}

\subsubsection{Функциональные требования к программному продукту}
На основании анализа предметной области в разрабатываемой программно-информационной системе удаленного администрирования компьютеров организации должны быть реализованы следующие функции:
Для клиентской части программы:
\begin{itemize}
	\item добавление ip-адрес сервера;
	\item удаление ip-адрес сервера;
	\item подключение к серверу;
	\item открытие файловых систем компьютеров;
	\item получение и просмотр файловой системы клиента;
	\item получение и просмотр файловой системы сервера;
	\item отправка файла на сервер;
	\item получение файла от сервера;
	\item открытие консольного управления сервером;
	\item отправка команды на сервер;
	\item получение и просмотр ответа от сервера;
	\item отключение от сервера.
\end{itemize}
Для серверной части программы:
\begin{itemize}
	\item ожидание подключения клиента;
	\item ожидание запроса от клиента;
	\item выполнение запроса клиента;
	\item отправка ответа клиенту;
\end{itemize}

На рисунке \ref{fig:-use_case_diagram} представлены функциональные требования к системе в виде диаграммы прецедентов.
\begin{figure}
	\centering
	\includegraphics[width=1\linewidth]{"images/Диаграмма прецедентов"}
	\caption{Диаграмма прецедентов}
	\label{fig:-use_case_diagram}
\end{figure}

\subsubsection{Требования пользователя к интерфейсу в программном продукте}
Интерфейс программного продукта должен соответствовать следующим требованиям:
Для клиентской части программы должны быть реализованы следующие графические элементы:
\begin{itemize}
	\item Кнопка добавления ip-адрес;
	\item Кнопка удаления ip-адрес;
	\item Поле ввода ip-адрес;
	\item Выпадающий список доступных ip-адрес;
	\item Кнопка открытие файловых систем компьютеров;
	\item Поле просмотра файловой системы клиента;
	\item Поле просмотра файловой системы сервера;
	\item Кнопка отправки файла на сервер;
	\item Кнопка получения файла от сервера;
	\item Кнопка открытие консольного управления сервером;
	\item Поле ввода команды для отправки на сервер;
	\item Поле просмотра ответа от сервера;
	\item Кнопка кнопка подключения к серверу;
	\item Кнопка отключение от сервера.
\end{itemize}
Серверная часть программы должна представлять из себя службу.

На рисунках \ref{fig:-maket_one} - \ref{fig:-maket_two} представлены функциональные требования к системе в виде диаграммы прецедентов.
\begin{figure}
	\centering
	\includegraphics[width=1\linewidth]{"images/Макет окна(консоль)"}
	\caption{Макет окна с консолью}
	\label{fig:-maket_one}
\end{figure}
\begin{figure}
	\centering
	\includegraphics[width=1\linewidth]{"images/Макет окна(файлы)"}
	\caption{Макет окна с файловой системой}
	\label{fig:-maket_two}
\end{figure}

\subsubsection{Варианты использования программного продукта}
\paragraph{Добавление ip-адрес}
Заинтересованные лица и их требования: IT-администратор, который хочет добавить ip-адрес для дальнейшей возможности подключения к серверу.
Предусловие: Пользователь запускает клиентскую часть приложения.
Постусловие: Пользователь добавляет ip-адрес.
Основной успешный сценарий:
\begin{enumerate}
	\item Пользователь вводит ip-адрес в строку.
	\item Пользователь нажимает кнопку "Добавить".
	\item ip-адрес появляется в выпадающем списке доступных ip-адрес.
\end{enumerate}

\paragraph{Удаление ip-адреса}
Заинтересованные лица и их требования: IT-администратор, который хочет удалить ip-адрес.
Предусловие: Пользователь запускает клиентскую часть приложения, ip-адрес был ранее добавлен.
Постусловие: Пользователь удаляет ip-адрес.
Основной успешный сценарий:
\begin{enumerate}
	\item Пользователь вводит ip-адрес в строку.
	\item Пользователь нажимает кнопку "Удалить".
	\item ip-адрес пропадает из выпадающего списка доступных ip-адрес.
\end{enumerate}

\paragraph{Подключение к серверу}
Заинтересованные лица и их требования: IT-администратор, который хочет подключиться к серверу.
Предусловие: Пользователь запускает клиентскую часть приложения, ip-адрес был ранее добавлен, на сервере запущена серверная часть программы.
Постусловие: Пользователь подключается.
Основной успешный сценарий:
\begin{enumerate}
	\item Пользователь выбирает ip-адрес из выпадающего списка.
	\item Пользователь нажимает кнопку "Подключить".
	\item Система выводит сообщение об успешном подключении.
\end{enumerate}

\paragraph{Открытие файловых систем компьютеров}
Заинтересованные лица и их требования: IT-администратор, который хочет открыть файловую систему компьютеров.
Предусловие: Пользователь запускает клиентскую часть приложения, было осуществлено подключение серверу.
Постусловие: Система отображает поля для просмотра и взаимодействия с файловой системой компьютеров.
Основной успешный сценарий:
\begin{enumerate}
	\item Пользователь нажимает кнопку "Файлы".
	\item Система отображает поля для просмотра и взаимодействия с файловой системой компьютеров.
\end{enumerate}

\paragraph{Получение и просмотр файловой системы клиента}
Заинтересованные лица и их требования: IT-администратор, который хочет посмотреть свою файловую систему.
Предусловие: Пользователь запускает клиентскую часть приложения, была нажата кнопка "Файлы".
Постусловие: Система отображает файловую систему клиента.
Основной успешный сценарий:
\begin{enumerate}
	\item Система отображает файловую систему клиента.
	\item Пользователь может с ней взаимодействовать.
\end{enumerate}

\paragraph{Получение и просмотр файловой системы сервера}
Заинтересованные лица и их требования: IT-администратор, который хочет посмотреть файловую систему сервера.
Предусловие: Пользователь запускает клиентскую часть приложения,  было осуществлено подключение серверу, была нажата кнопка "Файлы".
Постусловие: Система отображает файловую систему клиента.
Основной успешный сценарий:
\begin{enumerate}
	\item Система делает запрос на получение файловой системы сервера.
	\item Система получает ответ и выводит его в поле.
	\item Пользователь может с ней взаимодействовать.
\end{enumerate}

\paragraph{Отправка файла}
Заинтересованные лица и их требования: IT-администратор, который хочет отправить файл на сервера.
Предусловие: Пользователь запускает клиентскую часть приложения,  было осуществлено подключение серверу, была нажата кнопка "Файлы".
Постусловие: Система отображает файловую систему клиента.
Основной успешный сценарий:
\begin{enumerate}
	\item Пользователь выбирает файл для отправки.
	\item Пользователь выбирает путь для получения.
	\item Пользователь нажимает кнопку "Отправить".
	\item Система отправляет файл.
\end{enumerate}

\paragraph{Получение файла}
Заинтересованные лица и их требования: IT-администратор, который хочет получить файл от сервера.
Предусловие: Пользователь запускает клиентскую часть приложения,  было осуществлено подключение серверу, была нажата кнопка "Файлы".
Постусловие: Система отображает файловую систему клиента.
Основной успешный сценарий:
\begin{enumerate}
	\item Пользователь выбирает файл для получения.
	\item Пользователь выбирает путь для сохранения.
	\item Пользователь нажимает кнопку "Получить".
	\item Система отправляет запрос на получение файла.
	\item Система получает файл.
\end{enumerate}

\paragraph{Открытие консольного управления сервером}
Заинтересованные лица и их требования: IT-администратор, который хочет открыть консольное управление сервером.
Предусловие: Пользователь запускает клиентскую часть приложения.
Постусловие: Система отображает поля для консольного управления сервером.
Основной успешный сценарий:
\begin{enumerate}
	\item Пользователь нажимает кнопку "Консоль".
	\item  Система отображает поля для консольного управления сервером.
\end{enumerate}

\paragraph{Отправка команды}
Заинтересованные лица и их требования: IT-администратор, который хочет отправить команду на сервера.
Предусловие: Пользователь запускает клиентскую часть приложения,  было осуществлено подключение серверу, была нажата кнопка "Консоль".
Постусловие: Система отправляет команду на сервер.
Основной успешный сценарий:
\begin{enumerate}
	\item Пользователь вводит для отправки.
	\item Пользователь нажимает Enter.
	\item Система отправляет команду.
\end{enumerate}

\paragraph{Получение ответа}
Заинтересованные лица и их требования: IT-администратор, который хочет отправить команду на сервера.
Предусловие: Пользователь запускает клиентскую часть приложения,  было осуществлено подключение серверу, была нажата кнопка "Консоль", была отправлена команда.
Постусловие: Система отправляет команду на сервер.
Основной успешный сценарий:
\begin{enumerate}
	\item Серверная часть приложения получает команду.
	\item Серверная часть выполняет команду в cmd.
	\item Серверная часть отправляет результат выполнения команды.
	\item Система получает и выводит ответ от сервера.
\end{enumerate}

\paragraph{Отключение от сервера}
Заинтересованные лица и их требования: IT-администратор, который хочет отключиться от сервера серверу.
Предусловие: Пользователь запускает клиентскую часть приложения, было осуществлено подключение серверу.
Постусловие: Пользователь отключается.
Основной успешный сценарий:
\begin{enumerate}
	\item Пользователь нажимает кнопку "Отключить".
	\item Система выводит сообщение об успешном отключении.
\end{enumerate}

\subsection{Нефункциональные требования к программному продукту}
\subsubsection{Требования к надежности}
В приложении не должно возникать критических ошибок, приводящих
к экстренному завершению работы.

\subsubsection{Требования к программному обеспечению}
Для реализации программной системы должен быть использован язык программирования высокого уровня С\#  для разработки поведения компонентов программы.

\subsubsection{Требования к оформлению документации}
Разработка программной документации и программного изделия должна производиться согласно ГОСТ 19.102-77 и ГОСТ 34.601-90. Единая система программной документации.