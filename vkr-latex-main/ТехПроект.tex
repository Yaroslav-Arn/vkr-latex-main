\newsection
\section{Технический проект}
\subsection{Общие сведения о программной систем}

Необходимо спроектировать и разработать программно-информационную систему для удаленного администрирования компьютеров организации.

Разрабатываемая программная система предназначена для использования IT-администраторами организации с целью повышения эффективности и качества их работы. А также оптимизации рабочего время пользователей и уменьшения затрат человеко-часов на решение возникших проблем.

Основной принцип работы системы заключается в установке устойчивого соединения между компьютерами IT-администратора и пользователя с последующей возможностью передачи данных. 

Целью разработки приложения удаленного администрирования компьютеров организации  является создание эффективного инструмента для повешения уровня обслуживания IT-администраторами компьютера пользователя.


\subsection{Обоснование выбора технологии проектирования}

Для разработки был выбран проект WinForms так как он предоставляет широкий спектр необходимых условий для успешного создания программного продукта в соответствии с требованиями пользователя:
\begin{enumerate}
	\item Простота и быстрота разработки: WinForms предоставляет интуитивно понятный инструментарий для создания пользовательского интерфейса. Благодаря этому, разработка приложения может быть выполнена быстро и эффективно, что особенно важно в условиях ограниченного времени на проект.

	\item Широкая поддержка в среде разработки: Технология WinForms широко поддерживается в среде разработки .NET, что обеспечивает доступ к богатому набору инструментов и ресурсов для разработчиков. Это включает в себя возможность использования Visual Studio, богатый выбор компонентов пользовательского интерфейса и библиотеки классов для работы с сетью и другими аспектами системы.

	\item Стабильность и надежность: WinForms является проверенной и стабильной технологией, которая широко используется в промышленных приложениях на протяжении многих лет. Это обеспечивает надежность работы приложения и минимизирует риск возникновения ошибок или проблем в процессе эксплуатации.

	\item Привычный пользовательский интерфейс: WinForms предлагает привычный пользовательский интерфейс для пользователей Windows, что может уменьшить время обучения и повысить удобство использования приложения.

	\item Поддержка Windows-ориентированных функций: WinForms хорошо поддерживает различные функции операционной системы Windows. Это позволяет легко интегрировать приложение с окружением Windows и использовать его функциональные возможности.
\end{enumerate}

\subsubsection{Описание используемых технологий и языков программирования}

В рамках разработки приложения удаленного администрирования компьютеров организации было принято решение использовать язык программирования C\# в сочетании с протоколом TCP/IP для обеспечения сетевого взаимодействия. Этот выбор обусловлен рядом факторов, которые обеспечивают эффективность, надежность и удобство в разработке и использовании приложения.

\paragraph{Язык программирования C\#}

C\# - это объектно-ориентированный язык программирования, разработанный корпорацией Microsoft. C\# является частью платформы .NET и широко используется для создания различных типов приложений, включая веб-приложения, мобильные приложения, приложения для настольных компьютеров и многое другое.

Особенности языка C\#, подходящие для данного проекта:
\begin{enumerate}
	\item Богатый набор стандартных библиотек: C\# предоставляет доступ к обширному набору стандартных библиотек .NET Framework, включая классы и компоненты для работы с сетью, управления потоками, обработки исключений и многого другого. Это позволяет разработчикам эффективно использовать готовые решения для реализации функциональности удаленного администрирования.

	\item Удобство работы с GUI: C\# обладает удобным синтаксисом и широкими возможностями для создания графического пользовательского интерфейса (GUI). Это особенно важно для разработки приложения с удобным и интуитивно понятным интерфейсом, который позволит администраторам эффективно управлять компьютерами организации.

	\item Поддержка многопоточности: C\# имеет встроенную поддержку многопоточности, что позволяет разработчикам создавать многопоточные приложения для эффективной работы с параллельными задачами, такими как обработка запросов от нескольких компьютеров одновременно.

	\item Высокая производительность: C\# компилируется в нативный машинный код с использованием JIT (Just-In-Time) компиляции, что обеспечивает высокую производительность выполнения кода. Это особенно важно для приложения удалённого администрирования, где требуется быстрый отклик на запросы администраторов.

	\item Поддержка современных парадигм программирования: C\# поддерживает современные парадигмы программирования, такие как асинхронное программирование, что позволяет разработчикам создавать эффективные и гибкие приложения с использованием современных подходов к разработке.
\end{enumerate}

\paragraph{Протокол TCP/IP}

Протокол TCP/IP (Transmission Control Protocol/Internet Protocol) является основным протоколом сетевого взаимодействия в интернете и локальных сетях. Он обеспечивает надёжную и универсальную передачу данных между компьютерами и сетевыми устройствами.

Особенности протокола TCP/IP, подходящие для данного проекта:
\begin{enumerate}
	\item Надежность передачи данных: TCP обеспечивает гарантированную доставку данных, контролируя потери, дублирование и порядок получения пакетов. Это особенно важно для приложения удаленного администрирования, где надежность передачи данных имеет высокий приоритет.

	\item Установка соединения: TCP устанавливает соединение между отправителем и получателем перед началом передачи данных, что обеспечивает надежность и порядок доставки. Это позволяет обеспечить безопасность и целостность передаваемой информации.

	\item Поддержка потоковой передачи данных: TCP обеспечивает передачу данных в виде потока байтов, что удобно для приложений, требующих непрерывного потока данных. В случае приложения удаленного администрирования, где требуется передача команд и ответов от удаленных компьютеров, потоковая передача данных является эффективным подходом.

	\item Универсальность и широкое распространение: Протокол TCP/IP является стандартом сетевого взаимодействия и широко поддерживается на различных платформах и устройствах. Это обеспечивает совместимость и возможность взаимодействия с различными компьютерами и сетевыми устройствами в организации.
\end{enumerate}

Использование протокола TCP/IP для сетевого взаимодействия в приложении удаленного администрирования компьютеров организации обеспечивает надежность, безопасность и эффективность передачи данных между администратором и пользователем.

\subsection{Проектирование архитектуры программной системы}
\subsubsection{Архитектура программной системы}
На рисунке \ref{fig:-class-diagramma} представлена диаграмма классов программной системы.
\begin{figure}
	\centering
	\includegraphics[width=0.9\linewidth]{"images/Диаграмма классов"}
	\caption{Диаграмма классов}
	\label{fig:-class-diagramma}
\end{figure}

\subsection{Проектирование пользовательского интерфейса}

На основе требований к пользовательскому интерфейсу, представленных в пункте 2.3.3 технического задания, был разработан интерфейс управления программной системой. Для разработки пользовательского интерфейса было выбрано использование компонентов формы в Winform.
 
Числами на рисунках обозначены номера объектов интерфейса.

На рисунках \ref{fig:-window-console} - \ref{fig:-window-file} представлен интерфейс программной системы.

\begin{figure}
	\centering
	\includegraphics[width=0.9\linewidth]{"images/Окно с консолью"}
	\caption{Окно с отображением консоли}
	\label{fig:-window-console}
\end{figure}

Элементами данного окна являются:
\begin{enumerate}
	\item Кнопка открытие файловых систем компьютеров;
	\item Кнопка открытие консольного управления сервером;
	\item Выпадающий список доступных Ip;
	\item Кнопка подключения к серверу;
	\item Кнопка отключения от сервера;
	\item Поле ввода Ip;
	\item Кнопка добавления Ip;
	\item Кнопка удаления Ip;
	\item Поле просмотра ответа от сервера;
	\item Поле ввода команды для отправки на сервер.
\end{enumerate}

\begin{figure}
	\centering
	\includegraphics[width=0.9\linewidth]{"images/Окно с файлами"}
	\caption{Окно с отображением файловой системы}
	\label{fig:-window-file}
\end{figure}

Элементами данного окна являются:
\begin{enumerate}
	\item Кнопка открытие файловых систем компьютеров;
	\item Кнопка открытие консольного управления сервером;
	\item Выпадающий список доступных Ip;
	\item Кнопка подключения к серверу;
	\item Кнопка отключения от сервера;
	\item Поле ввода Ip;
	\item Кнопка добавления Ip;
	\item Кнопка удаления Ip;
	\item Поле просмотра файловой системы клиента;
	\item Кнопка отправки файла на сервер;
	\item Кнопка получения файла от сервера;
	\item Поле просмотра файловой системы сервера.
\end{enumerate}
