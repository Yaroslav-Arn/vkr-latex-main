\newsection
\section{Технический проект}
\subsection{Общие сведения о программной систем}

Необходимо спроектировать и разработать программно-информационную систему для удаленного администрирования компьютеров организации.

Разрабатываемая программная система предназначена для использования IT-администраторами организации с целью повышения эффективности и качества их работы. А также оптимизации рабочего время пользователей и уменьшения затрат человеко-часов на решение возникших проблем.

Основной принцип работы системы заключается в установке устойчивого соединения между компьютерами IT-администратора и пользователя с последующей возможностью передачи данных. 

Целью разработки приложения удаленного администрирования компьютеров организации  является создание эффективного инструмента для повешения уровня обслуживания IT-администраторами компьютера пользователя.


\subsection{Обоснование выбора технологии проектирования}

Для разработки был выбран проект WinForms так как он предоставляет широкий спектр необходимых условий для успешного создания программного продукта в соответствии с требованиями пользователя:
\begin{enumerate}
	\item Простота и быстрота разработки: WinForms предоставляет интуитивно понятный инструментарий для создания пользовательского интерфейса. Благодаря этому, разработка приложения может быть выполнена быстро и эффективно, что особенно важно в условиях ограниченного времени на проект.

	\item Широкая поддержка в среде разработки: Технология WinForms широко поддерживается в среде разработки .NET, что обеспечивает доступ к богатому набору инструментов и ресурсов для разработчиков. Это включает в себя возможность использования Visual Studio, богатый выбор компонентов пользовательского интерфейса и библиотеки классов для работы с сетью и другими аспектами системы.

	\item Стабильность и надежность: WinForms является проверенной и стабильной технологией, которая широко используется в промышленных приложениях на протяжении многих лет. Это обеспечивает надежность работы приложения и минимизирует риск возникновения ошибок или проблем в процессе эксплуатации.

	\item Привычный пользовательский интерфейс: WinForms предлагает привычный пользовательский интерфейс для пользователей Windows, что может уменьшить время обучения и повысить удобство использования приложения.

	\item Поддержка Windows-ориентированных функций: WinForms хорошо поддерживает различные функции операционной системы Windows. Это позволяет легко интегрировать приложение с окружением Windows и использовать его функциональные возможности.
\end{enumerate}

\subsection{Описание используемых технологий и языков программирования}

В рамках разработки приложения удаленного администрирования компьютеров организации было принято решение использовать язык программирования C\# в сочетании с протоколом TCP/IP для обеспечения сетевого взаимодействия. Этот выбор обусловлен рядом факторов, которые обеспечивают эффективность, надежность и удобство в разработке и использовании приложения.

\subsubsection{Язык программирования C\#}

C\# - это объектно-ориентированный язык программирования, разработанный корпорацией Microsoft. C\# является частью платформы .NET и широко используется для создания различных типов приложений, включая веб-приложения, мобильные приложения, приложения для настольных компьютеров и многое другое.

Особенности языка C\#, подходящие для данного проекта:
\begin{enumerate}
	\item Богатый набор стандартных библиотек: C\# предоставляет доступ к обширному набору стандартных библиотек .NET Framework, включая классы и компоненты для работы с сетью, управления потоками, обработки исключений и многого другого. Это позволяет разработчикам эффективно использовать готовые решения для реализации функциональности удаленного администрирования.

	\item Удобство работы с GUI: C\# обладает удобным синтаксисом и широкими возможностями для создания графического пользовательского интерфейса (GUI). Это особенно важно для разработки приложения с удобным и интуитивно понятным интерфейсом, который позволит администраторам эффективно управлять компьютерами организации.

	\item Поддержка многопоточности: C\# имеет встроенную поддержку многопоточности, что позволяет разработчикам создавать многопоточные приложения для эффективной работы с параллельными задачами, такими как обработка запросов от нескольких компьютеров одновременно.

	\item Высокая производительность: C\# компилируется в нативный машинный код с использованием JIT (Just-In-Time) компиляции, что обеспечивает высокую производительность выполнения кода. Это особенно важно для приложения удаленного администрирования, где требуется быстрый отклик на запросы администраторов.

	\item Поддержка современных парадигм программирования: C\# поддерживает современные парадигмы программирования, такие как асинхронное программирование, что позволяет разработчикам создавать эффективные и гибкие приложения с использованием современных подходов к разработке.
\end{enumerate}

\subsubsection{Протокол TCP/IP}

Протокол TCP/IP (Transmission Control Protocol/Internet Protocol) является основным протоколом сетевого взаимодействия в интернете и локальных сетях. Он обеспечивает надёжную и универсальную передачу данных между компьютерами и сетевыми устройствами.

Особенности протокола TCP/IP, подходящие для данного проекта:
\begin{enumerate}
	\item Надежность передачи данных: TCP обеспечивает гарантированную доставку данных, контролируя потери, дублирование и порядок получения пакетов. Это особенно важно для приложения удаленного администрирования, где надежность передачи данных имеет высокий приоритет.

	\item Установка соединения: TCP устанавливает соединение между отправителем и получателем перед началом передачи данных, что обеспечивает надежность и порядок доставки. Это позволяет обеспечить безопасность и целостность передаваемой информации.

	\item Поддержка потоковой передачи данных: TCP обеспечивает передачу данных в виде потока байтов, что удобно для приложений, требующих непрерывного потока данных. В случае приложения удаленного администрирования, где требуется передача команд и ответов от удаленных компьютеров, потоковая передача данных является эффективным подходом.

	\item Универсальность и широкое распространение: Протокол TCP/IP является стандартом сетевого взаимодействия и широко поддерживается на различных платформах и устройствах. Это обеспечивает совместимость и возможность взаимодействия с различными компьютерами и сетевыми устройствами в организации.
\end{enumerate}

Использование протокола TCP/IP для сетевого взаимодействия в приложении удаленного администрирования компьютеров организации обеспечивает надежность, безопасность и эффективность передачи данных между администратором и пользователем.

\subsubsection{Дополнительные протоколы}
Исходя из информации полученной из предыдущего пункта мы видим, что протокол TCP имеет массу преимуществ, которое подходят нам для создания необходимой нам программной системы. Однако, несмотря на все его преимущества, в нашей конкретной задаче мы сталкиваемся с некоторыми трудностями, для решения которой нам нужны другие специфические протоколы.

Наша система предназначена для передачи файлов, выполнения команд и обработки запросов. В этом контексте TCP обеспечивает надежную и упорядоченную передачу данных, что критически важно для корректной доставки файлов и выполнения команд. Однако есть несколько аспектов, в которых TCP может не полностью удовлетворять наши требования.

Во-первых, при передаче файлов, особенно больших объемов данных, важно обеспечить безопасность и целостность передаваемой информации. Для этого мы будем использовать протокол FilePath.

Во-вторых, выполнение команд требует быстрого и надежного обмена сообщениями. TCP отлично подходит для этого, так как гарантирует доставку данных. Однако для гармоничного взаимодействия с другими протоколами необходим новый, функцию которого выполняет протокол Command.

В-третьих, обработка запросов требует гибкости и возможности работы с различными типами данных и форматами сообщений. Для реализации этой функциональности мы будем использовать протоколы DirectoryDrive, DirectoryFolder, ReceiveFile.

\paragraph{Протокол FilePath}
Протокол отправки файла на сервер предназначен для передачи файлов между клиентом ик сервером с использованием TCP-соединения. Механизм работы протокола осуществляется с помощью функций SendFileAsync и ReceiveFileAsync(для отправки и получения файла соответственно).

Метод SendFileAsync предназначен для отправки файла между клиентом и сервером. Он асинхронно передает указанный файл.

Параметрами функции выступают:
\begin{itemize}
	\item filePathClient: путь, который указывает расположение файла для отправки или папки для сохранения на стороне клиента.
	\item filePathServer: путь, который указывает расположение файла для отправки или папки для сохранения на стороне сервера.
\end{itemize}

Возвращаемое значение:
\begin{itemize}
	\item Метод не возвращает значения. Выполняется асинхронно.
\end{itemize}

Описание процесса:
\begin{enumerate}
	\item Константа с именем протокола преобразуется в массив байт и отправляется первой. Это означает что сейчас будет получен файл.
	\item Вычисляется путь для сохранения файла, преобразуется в массив байт и отправляется вторым.
	\item Вычисляется длинна файла, преобразуется в массив байт и отправляется третьей.
	\item Последним по указанному пути открывается файл и отправляется блоками массивов байт.
\end{enumerate}

Так как первые 3 пункта представляют строку, буфер для них представляет 4 байта. Для файла буфер(он же блок) составляет 32 Кбайт.

Метод ReceiveFileAsync предназначен для получения файла. Он сохраняет его по указанному пути.

Функция не имеет параметров и выходных данных.

Описание процесса:
\begin{enumerate}
	\item Считывает массив байт и конвертирует его в строку, по которой нужно сохранить файл.
	\item Считывает массив байт и преобразует его в длину файла и запускает цикл для считывания файла.
	\item Считывает блоками файл.
\end{enumerate}

\paragraph {Протокол Command}
Протокол отправки команды на сервер предназначен для передачи сообщений cmd между клиентом и сервером с использованием TCP-соединения. 

На стороне клиента механизм отправки команды представлен следующим образом:
\begin{enumerate}
	\item Константа с именем протокола преобразуется в массив байт и отправляется первой. Это означает что сейчас будет получена команда.
	\item Команда преобразуется в массив байт и отправляется на сервер
	\item Полученный ответ в режиме реального времени выводится на экран.
\end{enumerate}

На стороне сервера при получении при получении команды вызывается и используются методы ExecuteCommandAsync(входными данными для него служит строка команды, выходных данных нету) и SendCommandAsync(входными данными являются строка и поток данных). 

Описание процесса:
\begin{enumerate}
	\item Отправляет полученную команду в cmd на выполнение.
	\item Константа с именем протокола преобразуется в массив байт и отправляется первой. Это означает что сейчас будет получена команда.
	\item Полученный ответ преобразуется в массив байт и отправляется обратно клиенту.
\end{enumerate}

Для отправки каждого блока этого протокола используется буфер в 4 байта.

\paragraph{Протокол DirectoryDrive}
Протокол отправки запроса списка дисков на сервере предназначен для получения клиентом списка дисков сервера с использованием TCP-соединения. 

На стороне клиента механизм отправки команды представлен следующим образом:
\begin{enumerate}
	\item Константа с именем протокола преобразуется в массив байт и отправляется на сервер.
	\item Полученный ответ в режиме реального времени выводится на экран.
\end{enumerate}

Когда сервер получает данный запрос запрос запускается следующий процесс:
\begin{enumerate}
	\item Выполняется функция получения списка дисков ПК.
	\item Константа с именем протокола преобразуется в массив байт и отправляется первой. Это означает что сейчас будет получен список дисков.
	\item Полученный список преобразуется в массив байт и отправляется обратно клиенту.
\end{enumerate}

Для отправки каждого блока этого протокола используется буфер в 4 байта.

\paragraph{Протокол DirectoryFolder}
Протокол отправки запроса списка подпапок в конкретной директории на сервере предназначен для получения клиентом списка дисков сервера с использованием TCP-соединения. 

На стороне клиента механизм отправки команды представлен следующим образом:
\begin{enumerate}
	\item Константа с именем протокола преобразуется в массив байт и отправляется на сервер.
	\item Строка с указанием пути преобразуется в массив байт и отправляется на сервер.
	\item Полученный ответ в режиме реального времени выводится на экран.
\end{enumerate}

Когда сервер получает данный запрос запрос запускается следующий процесс:
\begin{enumerate}
	\item Выполняется функция получения списка подпапок в полученной директории ПК.
	\item Константа с именем протокола преобразуется в массив байт и отправляется первой. Это означает что сейчас будет получен список дисков.
	\item Полученный список преобразуется в массив байт и отправляется обратно клиенту.
\end{enumerate}

Для отправки каждого блока этого протокола используется буфер в 4 байта.

\paragraph{Протокол ReceiveFile}
Протокол отправки запроса списка подпапок в конкретной директории на сервере предназначен для получения клиентом списка дисков сервера с использованием TCP-соединения. 

На стороне клиента механизм отправки команды представлен следующим образом:
\begin{enumerate}
	\item Константа с именем протокола преобразуется в массив байт и отправляется на сервер.
	\item Строка с указанием путей для  указания файла и сохранения преобразуется в массив байт и отправляется на сервер.
	\item Полученный ответ обрабатывается как в протоколе FilePath.
\end{enumerate}

Когда сервер получает данный запрос запрос запускаются методы описанные в протоколе FilePath.

\subsection{Проектирование архитектуры программной системы}
\subsubsection{Архитектура программной системы}
\paragraph{Описание клиента}
Клиентская часть программы представляет собой проект WinForms. Эта часть программной системы непосредственно взаимодействует с пользователем, который представлен IT-администратором. Поэтому при разработке классов необходимо учитывать не только функциональные возможности, но и удобства при работе с интерфейсом.На основании этих требований была разработана диаграмма классов клиентской части, которая представлена на рисунке \ref{fig:-class-diagrama-client}.
\begin{figure}
	\centering
	\includegraphics[width=0.95\linewidth]{"images/Диаграмма клиента"}
	\caption{Диаграмма классов клиента}
	\label{fig:-class-diagrama-client}
\end{figure}

В клиентской части программы находятся следующие классы:
\begin{itemize}
	\item IniFile: для подключения к компьютеру пользователя используется ip-адрес. Так как компьютеров в организации много и вводить адрес каждый раз неудобно, было принято решение о добавлении в проект INI файла для хранения ip-адресов. Также для взаимодействия с этим файлом был создан этот класс. В нем содержатся функции которые позволяют пользователю сохранять и удалять ip-адреса ПК к которым нужно подключиться. Также все записи из файла автоматически выводятся в список для создания подключения. 
	\item Client: этот класс создает окно с интерфейсом для работы администратора. Помимо добавления компонентов в нем также осуществляется их настройка. То есть при изменении размера формы компоненты также будут менять свой размер пропорционально этому. Помимо этот класс напрямую взаимодействует с пользователем. В нем обрабатываются все события, которые вызываются формой: нажатие кнопок, заполнение полей, работа с компонентами treeView. 
	\item TcpSocket: этот класс отвечает за выполнения всех механик связанных с удаленным управлением. В нем осуществляется подключение к серверу, после которого осуществляются следующие действия: 
	\begin{enumerate}
		\item Отправка запроса - в данном классе имеется набор функций направленных на реализацию механики отправки запроса по соответствующему протоколу.
		\item Отправка команды - в данном классе имеется набор функций направленных на реализацию механики отправки команды по соответствующему протоколу.
		\item Отправка файла - в данном классе имеется набор функций направленных на реализацию механики считывания и отправки файла по соответствующему протоколу.
		\item Получение файла - в данном классе имеется набор функций, с помощью которого реализуется получение и сохранение файла от сервера по полученному пути.
		\item Получение ответа - в данном классе имеется набор функций, с помощью которого осуществляется получение ответа от сервера.
		\item Отключение от сервера - данный класс при необходимости обрывает текущее соединение с сервером.
	\end{enumerate}
	\item FileView: этот класс отвечает за обработку в форме файловой системы как ПК клиента, так и ПК сервера. Через него производится раскрытие и закрытие директории, просмотр её содержимого и выбор пути к необходимому файлу или папке.
\end{itemize}

\paragraph{Описание сервера}
Сервер представляет из себя службу, которая запускает на всех необходимых IT-администратору ПК. На рисунке \ref{fig:-class-diagrama-server} представлена диаграмма классов сервера.
\begin{figure}
	\centering
	\includegraphics[width=1\linewidth]{"images/Диаграмма сервера"}
	\caption{Диаграмма классов сервера}
	\label{fig:-class-diagrama-server}
\end{figure}

В серверной части программы находятся один класс Server, который обрабатывает и выполняет все входяще запросы в соответствии с протоколом в фоновом режиме. 

При включении службы запускается функция по ожиданию подключения. После установки соединения осуществляются следующие действия: 
\begin{enumerate}
	\item Получение запроса - в данном классе имеется набор функций направленных на реализацию механики получения запроса по соответствующему протоколу и его выполнение.
	\item Выполнение команды - в данном классе имеется набор функций направленных на реализацию механики выполнения команды в cmd и отправки данных обратно клиенту.
	\item Отправка файла - в данном классе имеется набор функций направленных на реализацию механики считывания и отправки файла по соответствующему протоколу.
	\item Получение файла - в данном классе имеется набор функций, с помощью которого реализуется получение и сохранение файла от клиента по полученному пути.
	\item Отправка списка дисков - в данном классе имеется набор функций, с помощью которого осуществляется считывание дисков на ПК и отправка их клиенту.
	\item Отправка содержимого папки или диска - в данном классе имеется набор функций, с помощью которого осуществляется считывание содержимого по полученному пути.
\end{enumerate}

\subsection{Проектирование пользовательского интерфейса}

На основе требований к пользовательскому интерфейсу, представленных в пункте 2.3.3 технического задания, был разработан интерфейс управления программной системой. Для разработки пользовательского интерфейса было выбрано использование компонентов формы в Winform.
 
Числами на рисунках обозначены номера объектов интерфейса.

На рисунках \ref{fig:-window-console} - \ref{fig:-window-file} представлен интерфейс программной системы.

\begin{figure}
	\centering
	\includegraphics[width=1\linewidth]{"images/Окно с консолью"}
	\caption{Окно с отображением консоли}
	\label{fig:-window-console}
\end{figure}

Элементами данного окна являются:
\begin{enumerate}
	\item Кнопка открытие файловых систем компьютеров;
	\item Кнопка открытие консольного управления сервером;
	\item Выпадающий список доступных ip-адрес;
	\item Кнопка подключения к серверу;
	\item Кнопка отключения от сервера;
	\item Поле ввода ip-адрес;
	\item Кнопка добавления ip-адрес;
	\item Кнопка удаления ip-адрес;
	\item Поле просмотра ответа от сервера;
	\item Поле ввода команды для отправки на сервер.
\end{enumerate}

\begin{figure}
	\centering
	\includegraphics[width=1\linewidth]{"images/Окно с файлами"}
	\caption{Окно с отображением файловой системы}
	\label{fig:-window-file}
\end{figure}

Элементами данного окна являются:
\begin{enumerate}
	\item Кнопка открытие файловых систем компьютеров;
	\item Кнопка открытие консольного управления сервером;
	\item Выпадающий список доступных ip-адрес;
	\item Кнопка подключения к серверу;
	\item Кнопка отключения от сервера;
	\item Поле ввода ip-адрес;
	\item Кнопка добавления ip-адрес;
	\item Кнопка удаления ip-адрес;
	\item Поле просмотра файловой системы клиента;
	\item Кнопка отправки файла на сервер;
	\item Кнопка получения файла от сервера;
	\item Поле просмотра файловой системы сервера.
\end{enumerate}
