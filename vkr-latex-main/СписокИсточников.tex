\newsection
\centertocsection{СПИСОК ИСПОЛЬЗОВАННЫХ ИСТОЧНИКОВ}

%\begin{hyphenrules}{nohyphenation} %отключение переноса слов в содержании
\begin{thebibliography}{21}
    \bibitem{csharp_book}
    Биллиг, В. А. Основы программирования на С\# / В.А. Биллиг. - М.: Интернет-университет информационных технологий, Бином. Лаборатория знаний, 2021. - 488 c.
    
    \bibitem{vagner_book}
    Вагнер, Билл С\# Эффективное программирование / Билл Вагнер. - М.: ЛОРИ, 2021. - 320 c.
    
    \bibitem{arch_book}
    Дейтел, П. Как программировать на Visual C\# 2012 / П. Дейтел. - М.: Питер, 2018. - 2180 c.
    
    \bibitem{csharp_eff}
    Культин, Н. С\# в задачах и примерах / Н. Культин. - М.: БХВ-Петербург, 2020. - 1293 c.
    
    \bibitem{charp_flenow}
    Фленов, Михаил Библия C\# / Михаил Фленов. - М.: БХВ-Петербург, 2021. - 560 c.
    
    \bibitem{arch_patterns}
    Албахари Дж. C\# 6.0. Справочник. Полное описание языка [Текст] / Дж . Албахари, Б. Албахари — 6-е изд. — Москва: Вильямc, 2016. — 1040 c.
    
    \bibitem{csharp_advanced}
    Зыков, С.В. Введение в теорию программирования. Объектно-ориентированный подход / С.В. Зыков. - 2-е изд., испр. - М. : Национальный Открытый Университет «ИНТУИТ», 2016. - 189 с. : схем. - (Основы информационных технологий). - Библиогр. в кн.
    
    \bibitem{csharp_cookbook}
    Кариев Ч.А. Разработка Windows-приложений на основе Visual C\# [Электронный ресурс] : учебное пособие / Ч.А. Кариев. — Электрон. текстовые данные. — Москва, Саратов: Интернет-Университет Информационных Технологий (ИНТУИТ), Вузовское образование, 2017. — 768 c.
    
    \bibitem{tcp_advanced}
    Microsoft TCP/IP. Учебный курс. - М.: Microsoft Press. Русская Редакция; Издание 3-е, испр., 2021. - 400 c.
    
    \bibitem{tcp_design}
    Microsoft TCP/IP: Учебный курс. - М.: Издательский отдел Русская редакция' ТОО 'Channel Trading Ltd., 2019. - 392 c.
    
    \bibitem{tcp_fate}
    Фейт, С. TCP / IP. Архитектура. Протоколы. Реализация / С. Фейт. - Москва: ИЛ, 2019. - 424 c.
    
    \bibitem{tcp_xant}
    Хант, К. TCP/IP. Сетевое администрирование / К. Хант. - М.: СПб: Символ-Плюс; Издание 3-е, 2018. - 816 c.
    
    \bibitem{tcp_device}
    Дэвис, Джозеф Microsoft Windows Server 2003. Протоколы и службы TCP/IP. Техническое руководство / Джозеф Дэвис , Томас Ли. - М.: Эком, 2019. - 752 c.
    
    \bibitem{tcp_parket}
    Паркер TCP/IP. Для профессионалов / Паркер, Сиян Тим; , К.. - М.: СПб: Питер; Издание 3-е, 2021. - 859 c.
    
    \bibitem{maklafine}
    Маклафлин, Б. Объектно-ориентированный анализ и проектирование / Б. Маклафлин, Г. Поллайс, Д. Уэст. - М.: Питер, 2017. - 857 c.
    
    \bibitem{oop}
    Приемы объектно-ориентированного проектирования: Паттерны проектирования / Э. Гамма и др. - М.: Addison Wesley Longman, Inc., 2019. - 368 c.
    
    \bibitem{stilmen}
    Стиллмен, Э. Изучаем C\# / Э. Стиллмен, Дж. Грин. - М.: Питер, 2017. - 688 c.
    
    \bibitem{grelyk}
    Грекул, В. И. Методические основы управления ИТ-проектами / В.И. Грекул, Н.Л. Коровкина, Ю.В. Куприянов. - М.: Интернет-университет информационных технологий, Бином. Лаборатория знаний, 2021. - 392 c.
    
    \bibitem{babaev}
    Бабаев С.И., Компьютерные сети. Часть 3. Стандарты и протоколы : учебник / С.И. Бабаев, Б.В. Костров, М.Б. Никифоров. — М.: КУРС, 2018. — 176 с.
    
    \bibitem{sergeev}
    Сергеев А.Н. Основы локальных компьютерных сетей: учебное пособие для СПО/ А.Н. Сергеев. – 3-е изд., стер. – Санкт-Петербург: Лань, 2023 . – 184 с.
\end{thebibliography}
%\end{hyphenrules}
