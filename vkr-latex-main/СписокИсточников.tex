\newsection
\centertocsection{СПИСОК ИСПОЛЬЗОВАННЫХ ИСТОЧНИКОВ}

%\begin{hyphenrules}{nohyphenation} %отключение переноса слов в содержании
\begin{thebibliography}{9}
    \bibitem{csharp_book}
    Троелсен, Э., Джеппесен, П. Программирование на языке C\# 7.0 и платформы .NET Core./ Э. Троелсен, П. Джеппесен.– М.: ДМК Пресс, 2018.– 960 с.– ISBN 978-5-97060-566-2.
    
    \bibitem{tcp_book}
    Лав, Э., Карнахан, Б. Сетевое программирование в системе UNIX./ Э. Лав, Б. Карнахан.– М.: ДМК Пресс, 2018.– 832 с.– ISBN 978-5-97060-571-6.
    
    \bibitem{arch_book}
    Фаулер, М. Рефакторинг: Улучшение существующего кода./ М. Фаулер.– СПб.: Питер, 2018.– 576 с.– ISBN 978-5-496-02621-5.
    
    \bibitem{csharp_eff}
    Хёрст, Г., Альбахари, Дж. Эффективное программирование на C\#./ Г. Хёрст, Дж. Альбахари.– Пер. с англ. – СПб.: Символ-Плюс, 2019.– 768 с.~– ISBN 978-5-9903398-2-1.
    
    \bibitem{arch_patterns}
    Гамма, Э. Приемы объектно-ориентированного проектирования. Паттерны проектирования./ Э. Гамма, Р. Хелм, Р. Джонсон, Д. Влиссидес.– М.: Питер, 2018.– 400 с.– ISBN 978-5-496-02723-6.
    
    \bibitem{csharp_advanced}
    Костарев, Д. С\#, .NET и платформа Microsoft./ Д. Костарев.– М.: БХВ-Петербург, 2016.– 528 с.– ISBN 978-5-9775-2005-9.
    
    \bibitem{tcp_advanced}
    Комаровский, В. Сокеты программирование в примерах./ В. Комаровский.– М.: Издательский дом <<Питер>>, 2019.– 432 с.– ISBN 978-5-9915-1072-7.
    
    \bibitem{arch_design}
    Эванс, Э. Язык шаблонов доменной модели./ Э. Эванс.– СПб.: Питер, 2016.– 384 с.– ISBN 978-5-496-02554-6.
    
    \bibitem{csharp_cookbook}
    Сейтман, Дж. C\# 7.0 и .NET Core 2.0 – рецепты программирования./ Дж. Сейтман.– М.: ДМК Пресс, 2018.– 640 с.– ISBN 978-5-97060-596-9.
    
\end{thebibliography}
%\end{hyphenrules}
