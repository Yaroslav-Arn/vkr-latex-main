\newsection
\centertocsection{ЗАКЛЮЧЕНИЕ}

Преимущества технологий удаленного администрирования заключается в возможности без физического присутствия эффективно управлять компьютерами на больших расстояниях.
  
Компании, видя, как развиваются информационные технологии, пытаются использовать их выгодно для своего бизнеса, разрабатывая свои программные продукты для того, чтобы оптимизировать процесс устранения неполадок, а также минимизировать затраты времени на устранения возникших проблем. 
Для повышения эффективности работы IT-администраторов была разработан программно-информационная система для удаленного администрирования компьютеров организации.

Основные результаты работы:

\begin{enumerate}
\item Проведен анализ предметной области. Выявлена необходимость использования TCP/IP.
\item Разработана концептуальная модель приложения. Разработана модель протоколов. Определены требования к системе.
\item Осуществлено проектирование приложения. Разработана архитектура серверной части. Разработана архитектура клиентской части.Разработан пользовательский интерфейс клиентской части.
\item Реализована и протестирована программная система. Проведено системное тестирование.
\end{enumerate}

Все требования, объявленные в техническом задании, были полностью реализованы, все задачи, поставленные в начале разработки проекта, были также решены.

Готовый рабочий проект представлен программно-информационной системой для удаленного администрирования компьютеров организации. Программный продукт используется в организации.
