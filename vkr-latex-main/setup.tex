\documentclass[a4paper,14pt]{article}
\usepackage{extsizes} % для размера 14
\usepackage[english, russian]{babel} % переносы
\usepackage{graphicx} % для вставки картинок
\usepackage{fontspec} % шрифт Times
\setmainfont{Times New Roman}
\usepackage{indentfirst} % отступ первого абзаца
\setlength{\parindent}{1.25cm} % отступ первой строки
%\usepackage[onehalfspacing]{setspace}% полуторный интервал
%\renewcommand{\baselinestretch}{1.5}
\def\onehalflinespread{1.41}
\linespread{\onehalflinespread}
\frenchspacing % пробел после двоеточия
\usepackage{geometry} % поля
\usepackage{float} % перенос текста после рисунков
\floatplacement{figure}{H}
\geometry{left=3cm}
\geometry{right=1.5cm}
\geometry{top=2cm}
\geometry{bottom=2cm}
\sloppy % увеличение отступов между словами в строке при невозможности переноса
\usepackage[defaultlines=2,all]{nowidow} % абзац с новой страницы при нехватке места на заданное количество строк
\renewcommand{\labelitemi}{-} % тире в ненумерованных списках

\graphicspath{{images/}} % путь к изображениям
\renewcommand{\thefigure}{\thesection.\arabic{figure}} % шаблон номера рисунков
\renewcommand{\thetable}{\thesection.\arabic{table}} % шаблон номера таблиц
\newcommand{\newsection}{\clearpage\setcounter{figure}{0}\setcounter{table}{0}} % начало нового раздела 
\newcommand{\continuecaption}[1]{\caption*{#1}\\ \hline }

\makeatletter
\newcommand{\centertocsection}{\@dblarg\ah@centertocsection}
\def\ah@centertocsection[#1]#2{\phantomsection\section*{#1}\def\\{\relax\ifhmode\unskip\fi\space\ignorespaces}\addcontentsline{toc}{section}{#2}}
\makeatother

%\usepackage{longtable}
\usepackage{xltabular}
\usepackage{makecell}
%\usepackage{setspace}
\renewcommand\theadfont{}
\newcolumntype{T}{>{\centering\arraybackslash}X} % новый тип столбца T - автоматическая ширина столбца с выравниванием по центру
\newcolumntype{R}{>{\raggedleft\arraybackslash}X} % новый тип столбца R - автоматическая ширина столбца с выравниванием по правому краю
\newcolumntype{C}[1]{>{\centering\let\newline\\\arraybackslash\hspace{0pt}}m{#1}} % новый тип столбца C - фиксированная ширина столбца с выравниванием по центру
\newcolumntype{r}[1]{>{\raggedleft\arraybackslash}p{#1}} % новый тип столбца r - фиксированная ширина столбца с выравниванием по правому краю
\newcommand{\centrow}{\centering\arraybackslash} % командой \centrow можно центрировать одну ячейку (заголовок) в столбце типа X или p, оставив в оcтальных ячейках другой тип выравнивания
\newcommand{\finishhead}{\endhead\hline\endlastfoot}
\usepackage{etoolbox}
\usepackage{totcount}
\newtotcounter{tablecnt}
\newtotcounter{figurecnt}
\newtotcounter{bibcnt}
\newtotcounter{imglistcnt}

\def\toppad{4mm}
\def\tablepreskip{17.5pt}
\def\tablebelowskip{8.5pt}
\def\figurepreskip{21.5pt}
\def\figureaboveskip{10pt}
\def\figurebelowskip{8.5pt}

\setlength{\LTpre}{\tablepreskip}
\setlength{\LTpost}{\tablebelowskip}

\newbool{processtable}
\setbool{processtable}{true}

\newbool{processfigure}
\setbool{processfigure}{true}

\AtBeginEnvironment{xltabular}{\ifbool{processtable}{\stepcounter{tablecnt}\renewcommand{\arraystretch}{\onehalflinespread}\linespread{1.0}\selectfont}{\vspace{-13pt}}}

\AtBeginEnvironment{longtable}{\stepcounter{tablecnt}\vspace{5pt}}
\AfterEndEnvironment{longtable}{\vspace{7pt}}

\usepackage[tableposition=top]{caption} % подпись таблицы вверху
\captionsetup{strut=off}
\setlength{\intextsep}{0pt} % Vertical space above & below [h] floats
\setlength{\textfloatsep}{0pt} % Vertical space below (above) [t] ([b]) floats
\DeclareCaptionLabelFormat{gostfigure}{Рисунок #2} %подпись рисунка
\DeclareCaptionLabelFormat{gosttable}{Таблица #2} %подпись таблицы
\DeclareCaptionLabelSeparator{gost}{~--~} %разделитель в рисунках и таблицах
\captionsetup{labelsep=gost}
\captionsetup[figure]{aboveskip=\figureaboveskip,belowskip=\figurebelowskip,justification=centering,labelformat=gostfigure}
\captionsetup[table]{font={stretch=\onehalflinespread},skip=0pt,belowskip=0pt,aboveskip=8.5pt,singlelinecheck=off,labelformat=gosttable}

\usepackage{titlesec}

\titleformat{name=\section,numberless}[block]
{\normalsize\filcenter}
{}
{0pt}{}  %настройка ненумерованного раздела (Введение, содержание)

\titleformat{\section}[block] %настройка заголовка раздела
{\normalsize\bfseries}
{\hspace{1.25cm}\thesection}
{1em}{}

\titleformat{\subsection}[block]  %настройка заголовка подраздела
{\normalsize\bfseries}
{\hspace{1.25cm}\thesubsection}
{1em}{}
\setlength\leftmarginii  {0pt}

\titleformat{\subsubsection}[block]
{\normalsize\bfseries}
{\hspace{1.25cm}\thesubsubsection}
{1em}{}

\titleformat{\paragraph}[block]
{\normalsize\bfseries}
{\hspace{1.25cm}\theparagraph}
{1em}{}

\titlespacing*{name=\section,numberless}{0pt}{\toppad}{\toppad} % отступы разделов
\titlespacing*{\section}{0pt}{\toppad}{\toppad}
\titlespacing*{\subsection}{0pt}{\toppad}{\toppad}
\titlespacing*{\subsubsection}{0pt}{\toppad}{\toppad}
\titlespacing*{\paragraph}{0pt}{\toppad}{\toppad}
\usepackage{array} % новый вид колонки с выравниванием по центру

\usepackage{enumitem}
\setlist{nolistsep,wide=\parindent,itemindent=*} % отступы вокруг списков, выравнивание с учетом разделителя

\usepackage[]{tocloft}
\setlength{\cftbeforetoctitleskip}{-1em}
\addto\captionsrussian{
	\renewcommand\contentsname{}
}
\renewcommand{\cftdot}{} % убрать отточие
\renewcommand{\cftsecfont}{\normalsize} % не жирный шрифт раздела в оглавлении
\renewcommand{\cftsecpagefont}{\textmd} % не жирный номер страницы для раздела
\setcounter{tocdepth}{4} % число уровней оглавления
\setcounter{secnumdepth}{4} % число уровней подразделов

\def\secwidth{0pt}
\setlength{\cftsecnumwidth}{\secwidth}
\setlength{\cftsubsecnumwidth}{\secwidth}
\setlength{\cftsubsubsecnumwidth}{\secwidth}
\setlength{\cftparanumwidth}{\secwidth}
\setlength{\cftbeforesecskip}{0pt} % интервал между разделами содержания
\setlength{\cftsecindent}{0pt}% Remove indent for \section
\setlength{\cftsubsecindent}{0pt}% Remove indent for \subsection
\setlength{\cftsubsubsecindent}{0pt}% Remove indent for \subsubsection
\setlength{\cftparaindent}{0pt}% Remove indent for \paragraph

\def\secindent{3.8em}
\renewcommand{\cftsecaftersnumb}{\hspace{\secindent}}
\renewcommand{\cftsubsecaftersnumb}{\hspace{\secindent}}
\renewcommand{\cftsubsubsecaftersnumb}{\hspace{\secindent}}
\renewcommand{\cftparaaftersnumb}{\hspace{\secindent}}

\makeatletter
\renewcommand\@biblabel[1]{#1.} %оформление элемента списка литературы
\renewenvironment{thebibliography}[1]
{\list{\@biblabel{\@arabic\c@enumiv}}%
	{\settowidth\labelwidth{\@biblabel{#1}}%
		\leftmargin\labelwidth
		\setlength{\topsep}{0pt}                  
		\setlength{\partopsep}{0pt}
		\setlength{\itemsep}{0pt}
		\setlength{\parsep}{0pt}
		\setlength{\parskip}{0pt}
		\setlength{\itemindent}{1.75cm}
		\setlength{\leftmargin}{0pt}
		\@openbib@code
		\usecounter{enumiv}%
		\let\p@enumiv\@empty
		\renewcommand\theenumiv{\@arabic\c@enumiv}}%
	\sloppy
	\clubpenalty4000
	\interlinepenalty10000 %запрет переноса элемента на новую страницу
	\@clubpenalty \clubpenalty
	\widowpenalty4000%
	\sfcode`\.\@m}
{\def\@noitemerr
	{\@latex@warning{Empty `thebibliography' environment}}%
	\endlist}

\newsavebox\saved@arstrutbox 
\newcommand*{\setarstrut}[1]{% команда для уменьшения высоты строки в таблице
	\noalign{%
		\begingroup
		\global\setbox\saved@arstrutbox\copy\@arstrutbox
		#1%
		\global\setbox\@arstrutbox\hbox{%
			\vrule \@height\arraystretch\ht\strutbox
			\@depth\arraystretch \dp\strutbox
			\@width\z@
		}%
		\endgroup
	}%
}
\newcommand*{\restorearstrut}{% восстановление высоты строки
	\noalign{%
		\global\setbox\@arstrutbox\copy\saved@arstrutbox
	}%
}

\makeatother

% Раскомментировать блок при необходимости использования страниц с альбомной ориентацией. Предпочтительнее их не использовать, а осуществлять поворт рисунков на 90 градусов.
%
%Пример использования:
%
%%\begin{albumpage}
%%	\begin{figure}
%	%		\centering
%	%%		\vskip 1.0cm %способ добавить отступ, если необходимо
%	%		\includegraphics[height=0.95\textheight]{img/precedents_common}
%	%		\caption{Диаграмма прецедентов}
%	%		\label{fig:precedentscommon}
%	%	\end{figure}
%%\end{albumpage}
%
%\newbool{onalbumpage}
%\setbool{onalbumpage}{false}
%\BeforeBeginEnvironment{figure}{%
%	\ifbool{onalbumpage}{% true part:
%	}{
%		\stepcounter{figurecnt}\addvspace{\figurepreskip}
%	}
%}
%
%\usepackage{lscape}
%\usepackage{pdflscape}
%\usepackage{everypage}
%
%\newlength{\hfoot}
%\newlength{\vfoot}
%\AddEverypageHook{\ifdim\textwidth=\linewidth\relax
%	\else\setlength{\hfoot}{-\topmargin}%
%	\addtolength{\hfoot}{-\headheight}%
%	\addtolength{\hfoot}{-\headsep}%
%	\addtolength{\hfoot}{-.5\linewidth}%
%	\ifodd\value{page}\setlength{\vfoot}{\oddsidemargin}%
%	\else\setlength{\vfoot}{\evensidemargin}\fi%
%	\addtolength{\vfoot}{\textheight}%
%	\addtolength{\vfoot}{\footskip}%
%	\raisebox{\hfoot}[0pt][0pt]{\rlap{\hspace{.981\vfoot}\rotatebox[origin=cB]{90}{\thepage}}}\fi}
%
%\newenvironment{albumpage}{
%	\captionsetup[figure]{belowskip=0pt}
%	\setbool{onalbumpage}{true}
%%	\newgeometry{left=2cm,right=2cm,top=1.5cm,bottom=3cm}
%	\begin{landscape}
%		\pagestyle{empty}
%}{
%	\end{landscape}
%	\pagestyle{plain}
%%	\restoregeometry
%	\setbool{onalbumpage}{false}
%	\captionsetup[figure]{belowskip=\figurebelowskip}
%}
%
% Закомментировать строку ниже
%
\BeforeBeginEnvironment{figure}{\ifbool{processfigure}{\stepcounter{figurecnt}\vspace{\figurepreskip}}{}}

\def\oldbibitem{} \let\oldbibitem=\bibitem
\def\bibitem{\stepcounter{bibcnt}\oldbibitem}

\newcommand{\fillcenter}[2][\fill]{%
	\lhrulefill{#1}%
	\makebox[0pt]{#2}\lhrulefill{#1}}
	
\newcommand{\fillleft}[2][\fill]{%
	\underline{#2}\lhrulefill{#1}}
	
\newcommand{\lhrulefill}[1]{%
	\leavevmode
	\leaders\hrule height -.67ex depth \dimexpr .67ex+.4pt\relax % define the leader
	\hskip\glueexpr#1/2\relax\relax % how much it should extend
	\kern0pt
}

\regtotcounter{page}

% работа содержит \formbytotal{tablecnt}{рисун}{ок}{ка}{ков} % основа слова, окончания для согласования с 1 (рисун_ок), 2  (рисун_ка) и 5 (рисун_ков)
\def\formbytotal#1#2#3#4#5{%
	\newcount\c
	\c \totvalue{#1}\relax
	\newcount\last
	\newcount\pnul
	\last \c\relax
	\divide \last 10
	\pnul \last\relax
	\divide\pnul 10
	\multiply \pnul-10
	\advance \pnul\last
	\multiply \last-10
	\advance \last\c
	\total{#1}~#2%
	\ifnum\pnul=1#5\else%
	\ifcase\last#5\or#3\or#4\or#4\or#4\else#5\fi
	\fi
}

%\usepackage{showframe} % show border
\usepackage[hidelinks]{hyperref}

\usepackage{color} %% это для отображения цвета в коде
\usepackage{listings} %% листинги кода

\setmonofont[Scale=0.7]{dejavu-sans-mono.ttf} % моноширный шрифт для листинга

\definecolor{codegreen}{rgb}{0,0.6,0}
\definecolor{codegray}{rgb}{0.5,0.5,0.5}
\definecolor{codepurple}{rgb}{0.58,0,0.82}

\lstset{ %
language=C,                 % выбор языка для подсветки (здесь это С)
numbers=left,               % где поставить нумерацию строк (слева\справа)
numberstyle=\tiny,           % размер шрифта для номеров строк
stepnumber=1,                   % размер шага между двумя номерами строк
numbersep=5pt,                % как далеко отстоят номера строк от подсвечиваемого кода
commentstyle=\color{codegreen},
keywordstyle=\color{magenta},
numberstyle=\tiny\color{codegray},
stringstyle=\color{codepurple},
basicstyle=\linespread{0.95}\ttfamily,
backgroundcolor=\color{white}, % цвет фона подсветки - используем \usepackage{color}
showspaces=false,            % показывать или нет пробелы специальными отступами
showstringspaces=false,      % показывать или нет пробелы в строках
showtabs=false,             % показывать или нет табуляцию в строках
frame=single,              % рисовать рамку вокруг кода
tabsize=2,                 % размер табуляции по умолчанию равен 2 пробелам
captionpos=t,              % позиция заголовка вверху [t] или внизу [b] 
breaklines=true,           % автоматически переносить строки (да\нет)
breakatwhitespace=false, % переносить строки только если есть пробел
escapeinside={\%*}{*)}   % если нужно добавить комментарии в коде
}

\makeatletter % чтобы допускались русские комментарии в листингах
\lst@InputCatcodes
\def\lst@DefEC{%
 \lst@CCECUse \lst@ProcessLetter
  ^^80^^81^^82^^83^^84^^85^^86^^87^^88^^89^^8a^^8b^^8c^^8d^^8e^^8f%
  ^^90^^91^^92^^93^^94^^95^^96^^97^^98^^99^^9a^^9b^^9c^^9d^^9e^^9f%
  ^^a0^^a1^^a2^^a3^^a4^^a5^^a6^^a7^^a8^^a9^^aa^^ab^^ac^^ad^^ae^^af%
  ^^b0^^b1^^b2^^b3^^b4^^b5^^b6^^b7^^b8^^b9^^ba^^bb^^bc^^bd^^be^^bf%
  ^^c0^^c1^^c2^^c3^^c4^^c5^^c6^^c7^^c8^^c9^^ca^^cb^^cc^^cd^^ce^^cf%
  ^^d0^^d1^^d2^^d3^^d4^^d5^^d6^^d7^^d8^^d9^^da^^db^^dc^^dd^^de^^df%
  ^^e0^^e1^^e2^^e3^^e4^^e5^^e6^^e7^^e8^^e9^^ea^^eb^^ec^^ed^^ee^^ef%
  ^^f0^^f1^^f2^^f3^^f4^^f5^^f6^^f7^^f8^^f9^^fa^^fb^^fc^^fd^^fe^^ff%
  ^^^^20ac^^^^0153^^^^0152%
  % Basic Cyrillic alphabet coverage
  ^^^^0410^^^^0411^^^^0412^^^^0413^^^^0414^^^^0415^^^^0416^^^^0417%
  ^^^^0418^^^^0419^^^^041a^^^^041b^^^^041c^^^^041d^^^^041e^^^^041f%
  ^^^^0420^^^^0421^^^^0422^^^^0423^^^^0424^^^^0425^^^^0426^^^^0427%
  ^^^^0428^^^^0429^^^^042a^^^^042b^^^^042c^^^^042d^^^^042e^^^^042f%
  ^^^^0430^^^^0431^^^^0432^^^^0433^^^^0434^^^^0435^^^^0436^^^^0437%
  ^^^^0438^^^^0439^^^^043a^^^^043b^^^^043c^^^^043d^^^^043e^^^^043f%
  ^^^^0440^^^^0441^^^^0442^^^^0443^^^^0444^^^^0445^^^^0446^^^^0447%
  ^^^^0448^^^^0449^^^^044a^^^^044b^^^^044c^^^^044d^^^^044e^^^^044f%
  ^^^^0401^^^^0451%
  %%%
  ^^00}
\lst@RestoreCatcodes
\makeatother

\usepackage{adjustbox} % поворот изображений