\newsection
\section*{РЕФЕРАТ}

Объем работы равен \formbytotal{page}{страниц}{е}{ам}{ам}. Работа содержит \formbytotal{figurecnt}{иллюстраци}{ю}{и}{й}, \formbytotal{tablecnt}{таблиц}{у}{ы}{}, \formbytotal{bibcnt}{библиографическ}{ий источник}{их источника}{их источников} и \formbytotal{imglistcnt}{лист}{}{а}{ов} графического материала. Количество приложений – 2. Графический материал представлен в приложении А. Фрагменты исходного кода представлены в приложении Б.

Перечень ключевых слов: система, организация, компьютер, IT- администратор, сервер, клиент, соединение, TCP, команда, ответ, сообщение.

Объектом разработки является приложение удаленного администрирования компьютеров организации.

Целью выпускной квалификационной работы является \linebreak повышения эффективности работы IT-администраторов организации с помощью приложения удаленного администрирования компьютеров.

В процессе создания приложения были выделены основные сущности путем создания информационных блоков, использованы классы и методы модулей, обеспечивающие работу с сущностями предметной области, разработаны разделы, содержащие информацию о компании.

При разработке приложения использовалась система транспортировки данных TCP/IP.

Разработанное приложение было успешно внедрено в компанию.
\newpage
\selectlanguage{english}
\section*{ABSTRACT}
  
The volume of work is \formbytotal{page}{page}{}{s}{s}. The work contains \formbytotal{figurecnt}{illustration}{}{s}{s}, \formbytotal{tablecnt}{table}{}{s}{s}, \formbytotal{bibcnt}{bibliographic source}{}{s}{s} and \formbytotal{imglistcnt}{sheet}{}{s}{s} of graphic material. The number of applications is 2. The graphic material is presented in annex A. The fragment of the source code is provided in annex B.

The list of keywords: system, organization, computer, IT administrator, server, client, connection, TCP, command, response, message.

The object of development is an application for remote administration of the organization's computers.

The purpose of the final qualification work is \linebreak to increase the efficiency of the organization's IT administrators using a remote computer administration application.

In the process of creating the application, the main entities were identified by creating information blocks, classes and methods of modules were used to work with the entities of the subject area, sections containing information about the company were developed.

During the development of the application, a TCP/IP data transport system was used.

The developed application was successfully implemented in the company.
\selectlanguage{russian}
